\documentclass{article}

% if you need to pass options to natbib, use, e.g.:
%     \PassOptionsToPackage{numbers, compress}{natbib}
% before loading neurips_2020

% ready for submission
% \usepackage{neurips_2020}

% to compile a preprint version, e.g., for submission to arXiv, add add the
% [preprint] option:
%     \usepackage[preprint]{neurips_2020}

% to compile a camera-ready version, add the [final] option, e.g.:
%     \usepackage[final]{neurips_2020}

% to avoid loading the natbib package, add option nonatbib:
     \usepackage[nonatbib]{neurips_2020}

\usepackage[utf8]{inputenc} % allow utf-8 input
\usepackage[T1]{fontenc}    % use 8-bit T1 fonts
\usepackage{hyperref}       % hyperlinks
\usepackage{url}            % simple URL typesetting
\usepackage{booktabs}       % professional-quality tables
\usepackage{amsfonts}       % blackboard math symbols
\usepackage{nicefrac}       % compact symbols for 1/2, etc.
\usepackage{microtype}      % microtypography

\title{Project Proposal - ECE 176}

% The \author macro works with any number of authors. There are two commands
% used to separate the names and addresses of multiple authors: \And and \AND.
%
% Using \And between authors leaves it to LaTeX to determine where to break the
% lines. Using \AND forces a line break at that point. So, if LaTeX puts 3 of 4
% authors names on the first line, and the last on the second line, try using
% \AND instead of \And before the third author name.

\author{%
  Name \\
  Department\\
  PID\\
  % examples of more authors
  \And
  Name2 \\
  Department2 \\
  PID2 \\
}

\begin{document}

\maketitle

\begin{abstract}
    Describe the overall picture of your project here. Give a short description about the different parts you are going to cover in the proposal, but do not include too many details here.
\end{abstract}

\section{Problem Definition}

In this section, you need to describe the problem you want to solve. Besides You can include the following things(You do not have to cover all of them):
\begin{itemize}
    \item the motivation of solving this problem.
    \item Some key parts of the problem.
    \item Your understanding of the problem.
\end{itemize}

\section{Tentative Method}

In this section, you should describe your tentative method(the method you are planning to use). Besides, You can include the following things(You do not have to cover all of them): 

\begin{itemize}
    \item the detailed structure of your tentative method.
    \item the reason of choosing the tentative method.
    \item the strength of the chosen method.
\end{itemize}

\section{Experiments}

In this section, You should include the following things:

\begin{itemize}
    \item the datasets you are planning to use:
        \subitem the brief introduction of the dataset
        \subitem the data format
        \subitem other information related to your experiments
    \item the experiments are planning to perform, and the purpose of performing it.
\end{itemize}


\section*{References}

In this section, you should include reference to the work you mentioned. Any choice of citation style is acceptable as long as you are consistent.

\end{document}
